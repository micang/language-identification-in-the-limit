% v2-acmsmall-sample.tex, dated March 6 2012
% This is a sample file for ACM small trim journals
%
% Compilation using 'acmsmall.cls' - version 1.3 (March 2012), Aptara Inc.
% (c) 2010 Association for Computing Machinery (ACM)
%
% Questions/Suggestions/Feedback should be addressed to => "acmtexsupport@aptaracorp.com".
% Users can also go through the FAQs available on the journal's submission webpage.
%
% Steps to compile: latex, bibtex, latex latex
%
% For tracking purposes => this is v1.3 - March 2012

\documentclass[prodmode,acmtecs]{acmsmall} % Aptara syntax

\usepackage{amssymb}

% Package to generate and customize Algorithm as per ACM style
\usepackage[ruled]{algorithm2e}
\renewcommand{\algorithmcfname}{ALGORITHM}
\SetAlFnt{\small}
\SetAlCapFnt{\small}
\SetAlCapNameFnt{\small}
\SetAlCapHSkip{0pt}
\IncMargin{-\parindent}

\usepackage{indentfirst}

% KK: commands for line breaks in table cells
\newcommand{\specialcell}[2][c]{%
  \begin{tabular}[#1]{@{}c@{}}#2\end{tabular}}
\newcommand{\specialcelll}[2][l]{%
  \begin{tabular}[#1]{@{}l@{}}#2\end{tabular}}

\usepackage{array}
\newcolumntype{L}[1]{>{\raggedright\let\newline\\\arraybackslash\hspace{0pt}}m{#1}}
\newcolumntype{C}[1]{>{\centering\let\newline\\\arraybackslash\hspace{0pt}}m{#1}}
\newcolumntype{R}[1]{>{\raggedleft\let\newline\\\arraybackslash\hspace{0pt}}m{#1}}

\usepackage{mathtools}
% Metadata Information
%\acmVolume{9}
%\acmNumber{4}
%\acmArticle{39}
%\acmYear{2010}
%\acmMonth{3}

% Document starts
\begin{document}



% Page heads
\markboth{E. Mark Gold}{Language Identification in the Limit}

% Title portion
\title{Language Identification in the Limit}
\author{E. Mark Gold
\affil{The RAND Corporation}
}
% NOTE! Affiliations placed here should be for the institution where the
%       BULK of the research was done. If the author has gone to a new
%       institution, before publication, the (above) affiliation should NOT be changed.
%       The authors 'current' address may be given in the "Author's addresses:" block (below).
%       So for example, Mr. Abdelzaher, the bulk of the research was done at UIUC, and he is
%       currently affiliated with NASA.

\begin{abstract}
\indent Language learnability has been investigated. This refers to the following situation: A class of possible languages is specified, together with a method of presenting information to the learner about an unknown language, which is to be chosen from the class.
The question is now asked, ``Is the information sufficient to determine which of the possible languages is the unknown language?'' Many definitions of learnability are possible, but only the following is considered here: Time is quantized and has a finite starting time. At each time the learner receives a unit of information and is to make a guess as to the identity of the unknown language on the basis of the information received so far.
This process continues forever. The class of languages will be considered \textit{learnable} with respect to the specified method of information presentation if there is an algorithm that the learner can use to make his guesses, the algorithm having the following property: Given any language of the class, there is some finite time after which the guesses will all be the same and they will be correct.

In this preliminary investigation, a \textit{language} is taken to be a set of strings on some finite alphabet. The alphabet is the same for all languages of the class. Several variations of each of the following two basic methods of information presentation are investigated: A \textit{text} for a language generates the strings of the language in any order such that every string of the language occurs at least once. An \textit{informant} for a language tells whether a string is in the language, and chooses the strings in some order such that every string occurs at least once.

It was found that the class of context-sensitive languages is learnable from an informant, but that not even the class of regular languages is learnable from a text.
\end{abstract}

%\category{C.2.2}{Computer-Communication Networks}{Network Protocols}

%\terms{Design, Algorithms, Performance}

%\keywords{Wireless sensor networks, media access control,
%multi-channel, radio interference, time synchronization}

%\acmformat{Gang Zhou, Yafeng Wu, Ting Yan, Tian He, Chengdu Huang, John A. Stankovic, and Tarek F. Abdelzaher, 2010. A multifrequency MAC specially designed for  wireless sensor network applications.}

% At a minimum you need to supply the author names, year and a title.
% IMPORTANT:
% Full first names whenever they are known, surname last, followed by a period.
% In the case of two authors, 'and' is placed between them.
% In the case of three or more authors, the serial comma is used, that is, all author names
% except the last one but including the penultimate author's name are followed by a comma,
% and then 'and' is placed before the final author's name.
% If only first and middle initials are known, then each initial
% is followed by a period and they are separated by a space.
% The remaining information (journal title, volume, article number, date, etc.) is 'auto-generated'.

%\begin{bottomstuff}
%This work is supported by the National Science Foundation, under
%grant CNS-0435060, grant CCR-0325197 and grant EN-CS-0329609.

%Author's addresses: G. Zhou, Computer Science Department,
%College of William and Mary; Y. Wu  {and} J. A. Stankovic,
%Computer Science Department, University of Virginia; T. Yan,
%Eaton Innovation Center; T. He, Computer Science Department,
%University of Minnesota; C. Huang, Google; T. F. Abdelzaher,
%(Current address) NASA Ames Research Center, Moffett Field, %California 94035.
%\end{bottomstuff}

\maketitle

\section{MOTIVATION: TO SPEAK A LANGUAGE}
The study of language identification described here derives its motivation from artificial intelligence. The results and the methods used also have implications in computational linguistics, in particular the construction of discovery procedures, and in psycholinguistics, in particular the study of child learning. These implications are discussed in Section 4.

I wish to construct a precise model for the intuitive notion ``able to speak a language'' in order to be able to investigate theoretically how it
can be achieved artificially. Since we cannot explicitly write down the
rules of English which we require one to know before we say he can
``speak English,'' an artificial intelligence which is designed to Speak
English will have to learn its rules from implicit information. That is,
its information will consist of examples of the use of English and/or of
an informant who can state whether a given usage satisfies certain rules
of English, but cannot state these rules explicitly.

For the purpose of artificial intelligence, a model of the rules of usage
of natural languages must be general enough to include the rules which
do occur in existing natural languages. This is a lower bound on the
generality of an acceptable linguistic theory. On the other hand, the considerations of the last paragraph impose an upper bound on generality:
For any language which can be defined within the model there must
be a training program, consisting of implicit information, such that it
it possible to determine which of the definable languages is being presented.

Therefore this research program consists of the study of two subjects:
Linguistic structure and the learnability of these structures. This report describes the first step of this program. A very naive model of
language is assumed, namely, a language is taken to be a distinguished
set of strings. Such a language is too simple to do anything with (for
instance, to give information or to pose problems), but it has enough
structure to allow its learnability to be investigated as follows: Models
of information presentation are defined, and for each I ask ``For which
classes of languages does a learning algorithm exist?''

In the second step of this program (Gold, 1966), which will not be
discussed here, nontrivial models of the usages of language are constructed. The next step will be to return to learnability theory and determine whether reasonable training programs exist for linguistic structures of this type.

\section{LANGUAGE IDENTIFICATION MODELS}

Appendix II lists intuitive definitions of some of the terminology of
recursive theory used herein.
Let $A$ be a finite set (the alphabet of the languages to be considered)
and $\Sigma A$ represent the set of all finite strings of elements from $A$. 
$A$ is to be considered fixed throughout this paper. The results presented in
the next chapter are independent of the cardinality of $A$ so long as it
is not void. A language $L$ will signify any subset of $\Sigma A$. In an actual
language this may represent, for instance, the set of meaningful strings
of words.

A \textit{language learnability} model will signify the following triple:

\begin{enumerate}
    \item A \textit{definition of learnability}.
    \item A \textit{method of information presentation}.
    \item A \textit{naming relation} which assigns names (perhaps more than one) to
languages. The ``learner'' identifies a language by stating one of its
names. The names could be called grammars.
\end{enumerate}

Only one definition of learnability, which will be called identifiability
in the limit, will be considered here. Six alternative methods of information 
presentation and two alternative naming relations will be considered,
making a total of twelve models of language learnability. The definitions
will now be given, and the results are stated in Section 3. The proofs
are in Appendix I. The basic ideas behind the proofs are described in
Sections 7 and 9.

Time will be taken to be quantized and start at a finite time:
\[ t = 1, 2, \cdots \]
At each time t the learner is presented with a unit of information $i_t$
concerning the unknown language $L$. In any language learnability
model, the method of information presentation consists of assigning to
each $L$ a set of allowable training sequences, \mbox{$i_1$, $i_2$, $\cdots$}.

\textsc{Learnability}. At each time $t$ the learner is to make a guess $g_t$ of a
name of $L$ based on the information it has received through time $t$.
Thus the learner is a function $G$ which takes strings of units of information into names:
\[ g_t = G(i_1, \cdots, i_t) \]
$L$ will be said to be \textit{identified in the limit} if, after some finite time, the
guesses are all the same and are a name of $L$. A class of languages will
be called \textit{identifiable in the limit} with respect to a given language learnability model 
if there is an effective learner, i.e., an algorithm for making
guesses, with the following property: Given any language of the class and
given any allowable training sequence for this language, the language will
be identified in the limit.

For each of the 12 models of language learnability the following question 
has been investigated (the results are in the next Section): Which
classes of languages are identifiable in the limit? Note that identifiability (learnability) is a property of classes of languages, not of individual
languages.

In the case of identifiability in the limit the learner does not necessarily 
know when his guess is correct. He must go on processing information 
forever because there is always the possibility that information
will appear which will force him to change his guess. If the learner were
required to know when his answer is correct (this is equivalent to ``finite
identifiability'' defined in Section 6), then none of the classes of languages
investigated in the next chapter would be learnable in any of the
learnability models. My justification for studying identifiability in the
limit is this: A person does not know when he is speaking a language
correctly; there is always the possibility that he will find that his grammar
contains an error. But we can guarantee that a child will eventually
learn a natural language, even if it will not know when it is correct.

\textsc{Information Presentation}. Two basic methods of information
presentation will be considered, ``text'' and ``informant.'' Three variations 
of each will be defined.

A \textit{text} for $L$ is a sequence of strings $x_1$, $x_2$, $\cdots$ from $L$
such that every string of $L$ occurs at least once in the text. At time $t$ the
learner is presented $x_t$. Note that for any given language many texts are
possible. The three variations of this method of information presentation to be
considered are obtained by putting different restrictions on the class of 
allowed texts:
\begin{enumerate}
    \item \textit{Arbitrary Text}: $x_t$ may be any function of $t$.
    \item \textit{Recursive Text}: $x_t$ may be any recursive function of $t$.
    \item \textit{Primitive Recursive Text}: $x_t$ may be any primitive recursive function of $t$.
\end{enumerate}

An \textit{informant} for $L$ can tell the learner whether any string is an element of $L$, and does so at each time $t$ for some string $y_t$. Three types
of informant will be considered; these differ in how the $y_t$ are chosen:
\begin{enumerate}
    \item \textit{Arbitrary Informant}: $y_t$ may be any function of $t$ so long as every string of $\Sigma A$ occurs at least once.
    \item \textit{Methodical Informant}: An enumeration is assigned a priori to the strings of $\Sigma A$, and $y_t$ is taken to be the $t$th string of the enumeration.
    \item  \textit{Request Informant}: At time $t$ the learner chooses $y_t$ on the basis of information received so far.
\end{enumerate}

\textsc{Naming Relation}. Two naming relations will be considered, ``tester''
and ``generator.'' In both cases a name of a language, i.e., a grammar,
will be a Turing machine: A \textit{tester} for $L$ is a Turing machine which is
a decision procedure for $L$, that is, the Turing machine defines the function
from strings to natural numbers which has the value $1$ for strings
in $L$ and $0$ for strings not in $L$. A \textit{generator} for $L$ is a Turing
machine which generates $L$, that is, it defines a function from positive
integers to strings such that the range of this function is exactly $L$.
A tester exists iff $L$ is recursive and a generator exists iff $L$ is recursively enumerable.

Two \textit{language learnability models} will be called \textit{equivalent} 
if exactly the same classes of languages are identifiable in the limit with
respect to either model. 
Two \textit{naming relations} will be called \textit{equivalent} if, for
every method of information presentation, the two language learnability
models obtained by using these naming relations are equivalent. Similarly,
two \textit{methods of information presentation} will be called
\textit{equivalent} if every naming relation yields two equivalent language
learnability models.

Suppose two naming relations are effectively intertranslatable. That
is, suppose there is an algorithm for each of the naming relations which,
given a name of a language in this naming relation, would yield a name
of the language in the other. Then these are equivalent naming relations.

It is well known that it is possible to effectively translate from testers
to generators. Therefore, given any method of information presentation,
any class of languages which is tester-identifiable in the limit must also
be generator-identifiable in the limit. However, it is not possible to
effectively translate from generators to testers, even if we restrict 
ourselves to recursive languages for which both are defined. Therefore,
it is possible for a method of information presentation to exist such that
a class of languages is generator-identifiable in the limit but not
tester-identifiable in the limit. An example of this is given in the next Section.
This subject is discussed further in Section~11.

The three variations of information presentation by informant are
equivalent. They are defined separately only in order to make this point.

\section{LANGUAGE IDENTIFICATION RESULTS}

For every pair consisting of one of the 12 learnability models together
with one of the language classes listed in Table~\ref{tbl:one} it has been determined
whether the class of languages is identifiable in the limit. The language
classes are listed in descending order, i.e., each class is properly contained
in the class above it. The dividing lines between identifiable in the limit
and nonidentifiable in the limit are shown in the table. The classes of
languages below the dividing line shown for a given model of language
learnability are identifiable in the limit with respect to this model;
those above the dividing line are not. It is possible to represent the
results by means of dividing lines in this way because of the following
obvious facts: If a class of languages is identifiable in the limit with
respect to a given language learnability model, then the same holds
for any subclass; if a class is not identifiable in the limit, then the same
holds for any superclass.

\begin{table}
\tbl{DIVIDING LINES BETWEEN LEARNABILITY AND NONLEARNABILITY OF LANGUAGES\label{tbl:one}}{%
\begin{tabular}{L{3.5cm}L{4.5cm}}
\hline
Learnability model & Class of languages \\
\hline
\multicolumn{2}{l}{Anomalous text{$^a$} $\xrightarrow{\makebox[1.6cm]{}}$ } \\
& Recursive enumerable \\
& recursive \\
\multicolumn{2}{l}{Informant $\xrightarrow{\makebox[2.5cm]{}}$ } \\
& Primitive recursive \\
& Context-sensitive \\
& Context-free \\
& Regular \\
& Superfinite \\
\multicolumn{2}{l}{Text $\xrightarrow{\makebox[3.2cm]{}}$ } \\
& Finite cardinality languages \\
\hline
\end{tabular}}
\begin{tabnote}%
\tabnoteentry{$^a$}{Anomalous text refers to the use of the generator-naming 
relation and information presentation by means of primitive recursive text.}
\end{tabnote}%
\end{table}%

In the table, ``informant'' refers to any of the three variations of 
informant together with either the generator- or tester-naming relation.
That is, the same results have been obtained, so far, for each of the six
language learnability models which utilize an informant. Of the six
language learnability models which utilize a text for information 
presentation, five of them have given the same results, shown as ``text'' in the
table. The remaining model, shown as ``anomalous text,'' is primitive
recursive text with the generator-naming relation.

A \textit{super-finite class of languages} denotes any class which contains all
languages of finite cardinality and at least one of infinite cardinality.

The anomalous model using a text is of no practical interest, but
three noteworthy conclusions can be drawn from it: (1) It shows that
restrictions on the order of presentation of elements of the text can
greatly increase the learnability power of this method of information
presentation. (2) Note that the difference between a text and an 
informant is that a text only presents the learner with positive instances,
namely, elements of the language, whereas an informant presents both
positive and negative instances. Therefore, one would expect the 
informant to be more powerful. However, ``anomalous text'' is more
powerful than any of the ``informant'' models, which shows that one
must carefully consider the details of the learnability model. (3) ``Anomalous
text'' shows that the choice of naming relation can make a difference since, 
in this case, the generator-naming relation is far more
powerful than tester.

\section{IMPLICATIONS OF LANGUAGE LEARNABILITY RESULTS}

\textsc{To the Study of Child Learning of Language}. Recently, psycholinguists
have begun to study the acquisition of grammar by children
(e.g., McNeill, 1966). Those working in the field generally agree that
most children are rarely informed when they make grammatical errors,
and those that are informed take little heed. In other words, it is believed
that it is possible to learn the syntax of a natural language solely
from positive instances, i.e., a ``text.'' However, the results presented
in the last Section show that only the most trivial class of languages
considered is learnable (in the sense of identification in the limit) from
text, neglecting ``anomalous text.'' If one accepts identification in the
limit as a model of learnability, then this conflict must lead to at least
one of the following conclusions:

1. The class of possible natural languages is much smaller than one
would expect from our present models of syntax. That is, even if English 
is context-sensitive, it is not true that any context-sensitive language 
can occur naturally. Equivalently, we may say that the child
starts out with more information than that the language it will be presented 
is context-sensitive. In particular, the results on learnability
from text imply the following: The class of possible natural languages,
if it contains languages of infinite cardinality, cannot contain all languages
of finite cardinality.

2. The child receives negative instances by being corrected in a way
we do not recognize. If we can assume that the child receives both positive 
and negative instances, then it is being presented information by an
``informant.'' The class of primitive recursive languages, which includes
the class of context-sensitive languages, is identifiable in the limit from
an informant. The child may receive the equivalent of negative instances 
for the purpose of grammar acquisition when it does not get
the desired response to an utterance. It is difficult to interpret the actual
training program of a child in terms of the naive model of a language
assumed here.

3. There is an a priori restriction on the class of texts which can occur,
such as a restriction on the order of text presentation. The child may
learn that a certain string is not acceptable by the fact that it never
occurs in a certain context. This would constitute a negative instance.

\textsc{To Artificial Intelligence}. The training program of an artificial
intelligence can certainly include an informant, whether or not children
receive negative instances. Therefore, the results of Table I show that a
learning algorithm can be constructed for the identification of primitive
recursive predicates on strings, which probably include all the predicates
children learn. However, for the purpose of efficiency it is still of
significance to determine what additional information may be available
to children, either in the form of an a priori restriction on the class of
predicates which can occur in natural languages, or in the form of information
which can be obtained from the order of presentation of
naturally occurring texts.

\textsc{To the Construction of Discovery Procedures}. Attempts have
been made to construct an algorithm for automatically generating a
phrase structure grammar for a language solely by analyzing a text of
the language. One approach (Lamb, 1961) uses the ``distributional
analysis'' of Harris (1951, 1964) and Hockett (1958). Namely, one associates
phrases which are found to occur in the same context, thereby
defining phrase categories and simultaneously enlarging the set of contexts which can be considered equivalent; then one records how phrase
categories are constructed by concatenation of phrase categories.
Another approach which has been proposed (Solomonoff, 1964) uses
``identification by enumeration,'' which is defined in Section~7.

These attempts suggest the question, ``Is there enough information in
a text, even one of unlimited length, to allow the identification of a
context-free language?'' The results presented in Section 3 show that it is
impossible to construct a learning algorithm for the entire class of
context-free languages if the only information is an arbitrary text. If
one wishes to assume restrictions on the order of presentation of the
text, then a successful learning algorithm must be sensitive to the order
of the text. Thus, statistical approaches such as distributional analysis
are not suitable for this purpose. However, it would be useful to determine 
if there are interesting subclasses of the class of context-free languages 
which can be identified in the limit by either of these approaches.

\section{IDENTIFICATION OF FUNCTIONS AND BLACK BOXES}

This Section is a summary of the results of a previous paper (Gold,
1965) which is devoted to the learnability of two types of objects other
than languages:

\textsc{Time Function}. At each time $t$, a \textit{time function} produces an output, a
positive integer, which depends only on $t$. Formally, a time function is a
function of one variable which takes positive integers (time) into positive
integers (outputs).

\textsc{Black Box}. A black box has provision for an input at each time, as
well as an output. Each output is determined by the inputs that have
previously been applied to the black box. More precisely, let an \textit{alphabet}
here signify either a finite set with at least two elements, or else the set
of positive integers. Then a black box consists of the following triple:
An input alphabet $I$; an output alphabet $O$; a black box function $b$
which takes input strings into the output alphabet, thereby determining 
the output at time $t$: $o_t = b(i_1, \cdots , i_t)$.

Thus, a time function is a special case of a black box. In the case of a
time function, $o_t$ depends only on $t$ and not on a previous input string.
A time function can be described as a black box with a degenerate input
alphabet consisting of one element.

Throughout the study of black box learnability, $I$ and $O$ are to be
considered as fixed alphabets, i.e., $I$ and $O$ are chosen a priori, and all
black boxes are to use these two alphabets.

In the case of time function learnability the following situation is
studied. The learner observes the successive outputs of a time function
and is to guess what function it is observing; that is, the learner consists
of an identity guessing algorithm $G$ which yields a guess gt at each time
$t$ as to the identity of the time function, $g_t$ being determined by the
outputs which the time function has produced so far: $g_t = G(o_1 ,\cdots, o_t)$.

In the case of black box identification, the learner consists of an
experimenting algorithm $E$ as well as an identity guessing algorithm $G$.
$E$ determines the input which the learner will apply to the unknown
black box at any time as a function of the previous outputs of the black
box: $i_t = E(o_1, \cdots , o_{t-1})$. The identity guessing algorithm makes a
guess, at each time $t$, as to the identity of the black box: $g_t = G(o_1, \cdots , o_t)$.

It is too much to require the learner to identify a black box in the
sense of finding its identity at the beginning of the experiment, $t = 1$.
This is because, for instance, the black box may be such that the first
input which the learner applies to it may trap the black box in a subset
of its possible states, so that the learner will never be able to determine
what the behavior of the black box would have been if its first input had
been different. Therefore, only \textit{weak learnability} will be considered;
namely, the learner will be asked to predict, at each time, the future
behavior of the black box. That is, the learner is to guess the present
black box function, rather than that at $t = 1$.

Only one model of time function learnability and one of black box
learnability will be considered. The method of information presentation
for each model was described above. As in the models of language 
learnability, in both of these models ``learnability'' will signify
``identification in the limit.'' The naming relation will be the following: The names
of a time function, or of a black box (actually, its black box function),
will be taken to be those Turing machines which compute it.

Three classes each of time functions and of black boxes have been
considered. Table~\ref{tbl:two} shows which of these are identifiable in the limit.
As in Table~\ref{tbl:one}, the classes are listed in descending order in Table~\ref{tbl:two}.
Finite automata time functions denote ultimately periodic functions.

\begin{table}
\tbl{DIVIDING LINES BETWEEN LEARNABILITY AND NONLEARNABILITY FOR TIME FUNCTIONS AND BLACK BOXES\label{tbl:two}}{%
\begin{tabular}{L{3.5cm}L{3.5cm}}
\hline
Type of object & Class of objects \\
\hline
& Recursive \\
\multicolumn{2}{l}{Time functions $\xrightarrow{\makebox[1.9cm]{}}$ } \\
& Primitive recursive \\
\multicolumn{2}{l}{Black boxes $\xrightarrow{\makebox[2.3cm]{}}$ } \\
& Finite automata \\
\hline
\end{tabular}}
\end{table}%

\section{ABSTRACT MODEL OF IDENTIFICATION}

An \textit{identification situation} consists of the following three items:

1. A \textit{class $\Omega$ of objects}. One of the objects will be chosen, the learner
will be presented information about it, and the learner is to figure out
which one it is.

2. A method of information presentation. At each time $t$ the learner
receives a unit of information $i_t$ which is chosen from a set $I$. The \textit{method of information presentation} consists of specifying, for each $\omega \in \Omega$, which
sequences of units of information, $i_1$ , $i_2$, $\cdots$, are allowable. Let the set
of allowable sequences be designated $I^\infty(\omega)$.

3, A naming relation. The learner is to identify the unknown object
by finding one of its names. A \textit{naming relation} consists of a set $N$ of
names and a function $f$ which assigns an object to each name, $f$: $N\rightarrow \Omega$.

The identification problem is to determine whether there is a rule the
learner can use to accomplish the following: For any object $\omega \in \Omega$
and for any information sequence from $I^\infty(\omega)$, on the basis of that information sequence the rule will yield a name $n$ of $\omega$, that is, $f(n) = \omega$.
Three variations of the identification problem are the following, of
which only the first is considered in this paper.

\textit{Identification in the limit} has made some appearances previously in
the pattern recognition literature (e.g., Aizerman \textit{et al}., 1964). In this
case the learner is to guess a name of the unknown object at each time.
It is required that there be a finite time after which the guesses are all
the same and are correct.

\textit{Finite identification} is the type of identification problem usually
considered. It is best known in automata theory (e.g., Gill, 1961). In
finite identification, the learner is to stop the presentation of information
at some finite time when it thinks it has received enough, and state the
identity of the unknown object. This is not possible unless there is some
finite time at which the information distinguishes the unknown object.
That is, no other object satisfies the information.

\textit{Fixed-time identification}. In this case the information sequence stops
after some finite time which is specified a priori and which is independent 
of the object being described. The learner is to then state the identity
of the unknown object.

Saying that a class of objects is identifiable in the limit implies not
only that a suitable guessing function $G$ exists, but that it is effective;
that is, there exists an algorithm which computes it. The class of objects
will be called \textit{ineffectively identifiable in the limit} if a suitable $G$ exists,
regardless of whether it is effective. Note that whether a class of objects
is ineffectively identifiable in the limit does not depend on the naming
relation so long as every object has at least one name. This is because any
two naming relations are intertranslatable if we do not require translation
to be effective.

A11 identification situation will be said to satisfy the \textit{distinguishability condition} if the $I^\infty(\omega)$ are disjoint; that is, if there is no information quence which describes two different objects.

An identification situation will be said to satisfy the \textit{collapsing uncertainty condition} if the following holds: For any information string
$i_1 , \cdots , i_t$ , let $\Omega_t$ denote the set of those objects which agree with the
information received so far, i.e., those as such that $I^\infty(\omega)$ contains an
information sequence which begins $i_1 , \cdots , i_t$. For any information
sequence, the $\Omega_t$ will be a descending sequence. The collapsing
uncertainty condition requires that, for any object $\omega$ and any information
sequence of $I^\infty(\omega)$, the limit set of the $\Omega_t$ contains only $\omega$. That is, for
any $\omega'$ different from $\omega$ there is a time after which the information will
eliminate $\omega'$, namely $\omega' \notin \Omega_t$.

\section{METHODS OF IDENTIFICATION IN THE LIMIT}

\textit{Identification by enumeration} refers to the following guessing rule:
Enumerate the class of objects in any way, perhaps with repetitions.
That is, choose a function from the positive integers to the class of 
objects such that the range of the function is the entire class. At time $t$
guess the unknown object to be the first object of the enumeration which
agrees with the information received so far, i.e., which is in $\Omega_t$. This
guessing rule will be effective if the following two conditions hold: (1)
Given any information string $i_1 , \cdots , i_t$ and any positive integer $n$,
there is an effective method for determining whether the $n$th object of
the enumeration is in $\Omega_t$. (2) There is an effective method for finding a
name of the $n$th object of the enumeration.

To be precise, ``identification by enumeration'' refers to a class of
guessing rules, since there are many possible enumerations.

If we assume that $I$ is countable, then any class of objects which is
ineffectively identifiable in the limit must be countable. This is because
the domain of the guessing function $G$, namely, finite strings of elements
of $I$, is countable.

Henceforth, it will be assumed that $I$ and $\Omega$ are countable, and that
every object has at least one name.


\begin{theorem}
For ineffective identifiability in the limit, the distinguishability 
condition is necessary and the collapsing uncertainty condition is
sufficient. Indeed, the collapsing uncertainty condition implies that
identification by enumeration gives ineffective identification in the limit
for any enumeration. If $I^\infty(\omega)$ is countable for every $\omega$, then the distinguishability condition is sufficient for ineffective identifiability in the limit.
\end{theorem}


\begin{proof} Ineffective identifiability in the limit $\Rightarrow$ distinguishability.
If the distinguishability condition does not hold, then there is an 
allowable information sequence which describes two different objects, so that
it is impossible to know which is the unknown object.

Collapsing uncertainty $\Rightarrow$ identification by enumeration gives 
ineffective identification in the limit for any enumeration: The object to be
identified must occur somewhere in the enumeration. Let its first occurrence be at position $n$, There are at most $n — 1$ different objects before
this position in the enumeration. The collapsing uncertainty condition
implies that there is some finite time after which none of these prior
objects will agree with the information presented to the learner. After
that time the unknown object will be the first object of the enumeration
which satisfies the information received, and will therefore be correctly
guessed by the learner.

$I^\infty(\omega)$ is countable for every $\omega$, together with distinguishability$\Rightarrow$ ineffective identifiability in the limit. If we wish to identify the information
sequence, then the collapsing uncertainty condition always holds. Saying 
that $I^\infty(\omega)$ is countable for every $\omega$ is equivalent to saying that the
set of allowable information sequences is countable. In this case, 
identification by enumeration may be used to identify in the limit the 
information sequence being presented to the learner. The distinguishability
condition implies that one can translate (not necessarily effectively)
from information sequences to objects, and therefore to names of objects.
\end{proof}

Returning to the specific case of language identification, note that 
information presentation by informant satisfies the collapsing uncertainty
condition no matter what class of languages is considered. That is why
the class of primitive recursive languages is identifiable in the limit from
an informant; namely, an effective enumeration of the characteristic
functions of this class of languages exists thereby giving an effective
identification-by-enumeration guessing rule (see Theorem~I.4, Appendix~I).

Information presentation by text satisfies the distinguishability condition
for any class of languages, but it does not satisfy collapsing uncertainty 
for any class of languages which contains two languages such
that one is a subset of the other.

The following guessing algorithm shows that the class of languages of
finite cardinality is identifiable in the limit from an arbitrary text:
Guess the unknown language to consist solely of the strings generated so
far by the text (see Theorem I.6).

To see that the entire class of recursively enumerable languages is
identifiable in the limit from primitive recursive text using the generator-naming relation (see Theorem I.7), note that the class of primitive 
recursive texts is effectively enumerable. Therefore, the text can be
effectively identified in the limit using identification by enumeration. Since
the text is a generator for the language, this is all that is needed. This is
an example of the method of identification described at the end of the
proof of Theorem~7.1.

\section{INEFFECTIVE IDENTIFIABILITY IN THE LIMIT RESULTS}

\textsc{Language Identifiability}. If information presentation is by informant, 
then the collapsing uncertainty condition is satisfied, so that any
countable class of languages is ineffectively identifiable in the limit using
identification by enumeration. Of course, all the languages must have
names.

If information presentation is by recursive or primitive recursive text,
then, again, any countable class of languages is ineffectively identifiable
in the limit. This is because there are only a countable number of possible texts, so that the text can be identified in the limit by means of
identification by enumeration.

If information presentation is by arbitrary text, then the results for
ineffective identifiability in the limit are the same as for effective identifiability 
in the limit; namely, the class of languages of finite cardinality
is ineffectively identifiable in the limit, but every proper superclass is
not. This can be proved by the same methods used to prove Theorems
I.6 and I.8.

\textsc{Time Function Identifiability}. The method of information presentation
in the model of time function learnability satisfies the collapsing
uncertainty condition. Therefore, any countable class of time functions
is ineffectively identifiable in the limit.

\textsc{Black Box Identifiability}. Any countable class of black boxes is
ineffectively (weak) identifiable in the limit. This can be proved by the
same method used to prove Theorems 9 and 10 in Gold (1965).

\section{THE WEAKNESS OF TEXT}

It is of great interest to find why information presentation by text is
so weak and under what circumstances it becomes stronger. Therefore,
it is worthwhile to understand the method used in Theorems I.8 and
I.9 to prove that any class of languages containing all finite languages
and at least one infinite language is not identifiable in the limit from a
text in five out of six of the models using text.

The basic idea is proof by contradiction. Consider any proposed guessing
algorithm. It must identify any finite language correctly after a
finite amount of text. This makes it possible to construct a text for the
infinite language which will fool the learner into making a wrong guess
an infinite number of times as follows. The text ranges over successively
larger, finite subsets of the infinite language. At each stage it repeats
the elements of the current subset long enough to fool the learner.

Thus, the method of proof of the negative results concerning text
depends on the possibility of there being a huge amount of repetition
in the text. Perhaps this can be prevented by some reasonable probabilistic
assumption concerning the generation of the text. In this case one
would only require identification in the limit with probability one,
rather than for every allowed text.

I have been asked, ``If information presentation is by means of text,
why not guess the unknown language to be the simplest one which accepts
the text available?'' This is identification by enumeration. It is
instructive to see why it will not work for most interesting classes of
languages: The universal language (if it is in the class) will have some
finite complexity. If the unknown language is more complex, then the
guessing procedure being considered will always guess wrong, since the
universal language is consistent with any finite text. This follows from
the fact that, if $L$ is the unknown language and if $L' \supset L$, then $L'$ is consistent with any finite segment of any text for $L$. The problem with text
is that, if you guess too large a language, the text will never tell you that
you are wrong.

\section{LEARNING TIME}

Consider an identification situation which satisfies the collapsing
uncertainty condition. Choose an enumeration of the class of objects and
let $G_0$ be the identification-by-enumeration guessing rule which uses this
enumeration. At first sight, identification by enumeration appears to be
a naive approach to learning. However, it will be shown that $G_0$ is the
most efficient possible guessing rule with respect to learning time. This
holds even if ineffective guessing rules are allowed and if the enumeration
has duplications. This result is somewhat surprising in view of the fact
that there are many different identification-by-enumeration guessing
rules, obtained by using different enumerations. This means that none of
them is uniformly better than any other, in the sense defined below, for
the purpose of minimizing learning time.

Let $G$ be any guessing rule, $\omega$ be any element of the class $\Omega$ of objects
and $\tilde{\imath}$ any information sequence allowed for $\omega$. Define the learning time 
$\tau(G, \omega, \tilde{\imath})$ to be the first time such that at that time and all following times
all the guesses of $G$ as to the identity of $\omega$ will be the same and correct.
Define the learning time to be $\infty$ if no such time exists.

Let $G$ and $G'$ be any two guessing rules. $G$ will be said to be uniformly
faster than $G$, if the following two conditions hold: (1) Given any $\omega$ and
any allowable $\tilde{\imath}_0$ for $\omega$, then $G$ will identify $\omega$ at least as soon as $G'$ will
identify $\omega$, that is,
\[
\tau(G, \omega, \tilde{\imath}) \leqq \tau(G', \omega, \tilde{\imath}) 
\]
(2) There is some $\omega_0$ and an allowable to for $\tilde{\imath}_0$ for $\omega_0$ such that $G$ will identify $\omega_0$ than~$G'$:
\[
\tau(G, \omega_0, \tilde{\imath}_0) \leqq \tau(G', \omega_0, \tilde{\imath}_0)
\]

\begin{theorem} If $G_0$ is an identification-by-enumeralion guessing rule,
then there is no guessing rule uniformly faster than $G_0$.
\end{theorem}

\begin{proof} This is what has to be proved: Let $G$ be any guessing rule. If
there is an $\omega$ and an allowable $\tilde{\imath}$ for $\omega$ such that $G$ is faster than $G_0$, i.e.,
$\tau(G, \omega, \tilde{\imath}) < \tau(G_0, \omega, \tilde{\imath})$, then there is an $\omega '$, and an allowable $\tilde{\imath}'$ for $\omega '$
such that $G_0$ is faster than $G$, i.e., $\tau(G_0, \omega ', \tilde{\imath}') < \tau(G, \omega ', \tilde{\imath}')$.

$G_0$ is constructed in such a way that, once it guesses correctly, its
guesses never change. Therefore, if $G_0$, is presented with the object $\omega$ to
identify, by being given information sequence $\tilde{\imath}$, then $\tau(G_0, \omega, \tilde{\imath})$ is the first
time that $G_0$ guesses the identity of the unknown object to be $\omega$. At the
earlier time, $\tau(G_0, \omega, \tilde{\imath})$, $G_0$ must guess the name of some other object, say
$\omega '$. At any time that $G_0$ guesses the name of an object, that object must
agree with the information received so far. That is, at the time that $G_0$
guesses $\omega '$, there must be an allowable $\tilde{\imath}'$ for $\omega '$, such that $\tilde{\imath}'$ is the same as
$\tilde{\imath}$ up to this time. Thus, if $\omega '$ were the unknown object and $\tilde{\imath}'$  the 
information sequence, then at that time, namely $\tau(G, \omega, \tilde{\imath})$, $G$ and $G_0$ would make
the same guesses as they would if presented $\omega$ and $\tilde{\imath}$: $G_0$ would guess $\omega '$
and $G$ would guess $\omega$. That is, if presented with $\omega '$, and $\tilde{\imath}'$, $G_0$ would be
correct before $G$.
\end{proof}

Note that the proof of Theorem~9.1 remains valid even if $G_0$ does not
identify every object of the class in the limit and $G$ does. It is only
necessary that, for every finite initial subsequence of every allowable
information sequence, there exist an object in the enumeration which is consistent
with it.

\section{TRANSLATION FROM GENERATORS T0 TESTERS}

The purpose of this Section is to use the results on language identification in the limit presented in Section~3 to solve a problem in recursive
theory. In addition to the generator- and tester-naming relations, a third
method of assigning names to recursive sets, called domain generators,
is defined below. Suzuki (1959) has shown that there is no effective
method for going from recursive sets, described by domain generators,
to their complements, described in the same way. This result will here be
strengthened in two ways: It will be shown that there is no 2-recursive
(defined below) translation of this type, and that, rather than the entire
class of recursive sets, one can restrict one's consideration to any class of
sets which contains all finite sets and at least one infinite set without
changing this result. It has been pointed out to me by Norman Shapiro
that one can easily construct a 3-recursive translation from recursive
sets to their complements, using the domain generator-naming relation,
thus completely establishing the difficulty, in the Kleene hierarchy, of
this type of translation.

For the purpose of this Section it is desirable to think of languages as
sets of positive integers, rather than sets of strings. This may be 
accomplished by means of any recursive one-to-one correspondence between
the strings of $\Sigma A$ and the positive integers.

Let $Z_n(x)$ be the number-theoretic function of one variable defined
by the Turing machine whose G{\"o}del number is $n$. The two naming relations for languages are defined formally as follows.

\textit{Generator.} A \textit{generator} for $L$ is a positive integer $n$ such that $L$ is the
range of $Z_n(x)$.

\textit{Tester.} A \textit{tester} for $L$ is a positive integer $n$ such that $Z_n(x) = \chi_L(x)$,
where $\chi_L$ is the characteristic function of $L$.

It will be assumed in this Section that in all naming relations the
\textit{names are numbers}.

\textit{Translation.} Given two naming relations, $N_1$ and $N_2$, a \textit{translation} from
$N_1$ to $N_2$ is a partial, number-theoretic function $f(n)$ such that, if $n$ is a
name of an object in $N_1$ , then $f(n)$ is defined and is a name of that object
in $N_2$.

\textit{Limiting recursive function}. A partial, number-theoretic function $f(n)$
will be called \textit{limiting recursive} if there is a total recursive ``guessing
function'' $g(n, t)$ such that
\begin{equation}
\label{eq:10.1}
f(n) = lim_t~g(n, t), 
\end{equation}


where the limit of a sequence of positive integers is taken to be undefined
if the sequence is not constant after some finite point; otherwise the
limit is defined to be the value at which the sequence ultimately becomes
constant.

\begin{theorem} If a class of objects is identifiable in limit using some
method of information presentation and using naming relation $N_1$, and if
there is a limiting recursive translation from $N_1$ to naming relation $N_2$,
then the class of objects is identifiable in the limit using the same method of information presentation and $N_2$.
\end{theorem}

\begin{proof}
Let $f(n)$ be a limiting recursive translation from $N_1$ to $N_2$,
and $g(n, t)$ be a total recursive function such that Eq.~(\ref{eq:10.1}) holds, and
let $G_1$ be a suitable guessing rule using $N_1$. Then a suitable guessing rule
using $N_2$ is the following:

For a given information sequence, suppose that at time $t$ the guess made
by $G_1$ is $g_t$. Then, when using $N_2$, let the guess be $g(g_t , t)$. Call this
guessing rule $G_2$.

To see that $G_2$ is suitable note that, using $N_1$, there is a time $t_1$ after
which all the $g_t$ will equal a fixed value $g_0$ , which is correct in $N_1$ . By
Eq.~(\ref{eq:10.1}), there is a $t_2$ such that, for all $t \geqq t_2$,
\begin{equation}
f(g_0) = g(g_0, t0),
\end{equation}
Therefore, for all times greater than $t_1$ and $t_2$ , the guess of $G_2$ will be
$f(g_0)$, which is correct in $N_2$.
\end{proof}

Consider language learnability with information presentation by
means of primitive recursive text. The results shown in Table I differ for
the two naming relations, generator and tester. This leads to the following conclusion:

\begin{corollary}
\label{coro:11.1}
If $C$ is a class of languages which contains all finite
languages and at least one infinite recursive language, then there is no limiting recursive translation from testers for $C$ to generators.
\end{corollary}

In order to compare this result with that of Suzuki, it is necessary to
define two more naming relations for recursive languages:

\textit{Domain generator}. A \textit{domain generator} for $L$ is a positive integer $n$ such
that $L$ is the domain of $Z_n(x)$.

\textit{Anti-domain generator}. $n$ is an \textit{anti-domain generator} for $L$ if it is a
domain generator for the complement of $L$.

\textit{Representing predicate}. For any partial number-theoretic function
$f(n)$, its representing predicate $P n, m)$ will be defined to be true for just
those pairs $(n, m)$ such that $f(n)$ is defined and $f(n) = m$.

\textit{k-Recursive function}. A partial, ndmber-theoretic function will be called
\textit{k-recursive} if its representing predicate is k-r.e. (recursively enumerable)
in the Kleene hierarchy.

Note that the l-recursive functions are just the partial recursive
functions.

It is shown elsewhere (Gold, 1965) that the limiting recursive functions 
are the same as the 2-recursive functions.

The strengthening of Suzuki’s result described at the beginning of this
chapter can now be stated formally:

\begin{theorem}
\label{theo:11.2}
If $C$ is a class of languages which contains all finite
languages and at least one infinite recursive language, then there is no 
2-recursive translation from domain generators for $C$ to anti-domain generators.
\end{theorem}

\begin{proof}
Theorem~\ref{theo:11.2} follows from Corollary~\ref{coro:11.1} together with the
following facts which can be proved by standard methods:

There is a partial recursive translation from testers to domain generators.

Given both a domain generator and an anti-domain generator for a
recursive set, there is an effective procedure for finding a generator for it.

Composition of 2-recursive and recursive functions yields 2-recursive
functions.
\end{proof}


\section{INDUCTIVE INFERENCE}

Concerning inductive inference, philosophers often occupy themselves
with the following type of question: Suppose we are given a body of
information and a set of possible conclusions, from which we are to
choose one. Some of the conclusions are eliminated by the information.
The question is, of the conclusions which are consistent with the 
information, which is ``correct''?

If some sort of probability distribution is imposed on the set of 
conclusions, then the problem is meaningful. But if no basis for choosing
between the consistent conclusions is postulated a priori, then inductive
inference can do no more than state the set of consistent conclusions.

The difficulty with the inductive inference problem, when it is stated
this way, is that it asks, ``hat is the correct guess at a specific time
with a fixed amount of information?'' There is no basis for choosing
between possible guesses at a specific time. However, it is interesting
to study a guessing strategy. Now one can investigate the limiting behavior
of the guesses as successively larger bodies of information are
considered. This report is an example of such a study. Namely, in 
interesting identification problems, a learner cannot help but make errors
due to incomplete knowledge. But, using an ``identification in the limit''
guessing rule, a learner can guarantee that he will be wrong only a
finite number of times.

\bigskip
\noindent \textsc{Received}: February 20, 1967


%\nocite{r1}
%\nocite{r2}
%\nocite{r3}
%\nocite{r4}
%\nocite{r5}
%\nocite{r6}
%\nocite{r7}
%\nocite{r8}
%\nocite{r9}
%\nocite{r10}

%\bibliographystyle{ACM-Reference-Format-Journals}
%\bibliography{lan}





\end{document}
% End of v2-acmsmall-sample.tex (March 2012) - Gerry Murray, ACM


